\chapter{Zaključak i budući rad}
		
		Zadatak naše ekipe bio je razvoj web aplikacije za spašavanje nestalih osoba iz područja pogođenih nepogodama. Projekt je bio podijeljen u dvije ključne faze, s ciljem ostvarenja postavljenih zadataka.
		
		U prvoj fazi, koja je trajala od 16. listopada 2023. do 17. studenog 2023., naš tim se okupio kroz PROGl app sustav. Prvi sastanak poslužio je za upoznavanje s projektnim zadatkom, nakon čega je uslijedila precizna podjela zaduženja među članovima tima. Grupiranje u podtimove za frontend, backend i dokumentaciju pokazalo se kao ključno kako bi se efikasno obuhvatile različite aspekte projekta. Fokus prve faze bio je na temeljitom dokumentiranju zahtjeva aplikacije, uz izradu opisa i dijagrama obrazaca uporabe, sekvencijskih dijagrama, opisa, modela i dijagrama baze podataka, te dijagrama razreda. Ovi koraci bili su od neprocjenjive pomoći u drugoj fazi projekta prilikom implementacije funkcionalnosti, pružajući jasnu smjernicu i olakšavajući suradnju između podtimova.
		
		Druga faza projekta, od 4. prosinca 2023. do 19. siječnja 2024., bila je usmjerena na izradu implementacijskog rješenja. Unatoč tehničkim izazovima koji su proizašli iz nedostatka iskustva u izradi web aplikacija, tim se pokazao odlučnim i spretnim u rješavanju problema. Dodatni fokus bio je na završetku dokumentacije, uključujući dijagrame stanja, aktivnosti, komponenti i razmještaja, te evaluaciju programskog rješenja. Aktivno sudjelovanje na projektu donijelo je vrijedno iskustvo svim članovima tima, pruživši priliku za stjecanje novih vještina u izradi web aplikacija, upoznavanje s novim tehnologijama i alatima, te unaprijeđenje komunikacijskih i organizacijskih vještina.
		
		S obzirom na ostvarene rezultate, grupa je uglavnom zadovoljna krajnjim proizvodom, koji je većinom implementiran u skladu s planiranim funkcionalnostima. Otvorenost za buduće nadogradnje i unaprijeđenja aplikacije ostavlja prostor za kontinuirani razvoj, dok je cijeli projekt pridonio razvoju timskog duha i stvaranju temelja za uspješno suočavanje s izazovima sličnih projekata u budućnosti.
		
		\eject 